\documentclass[pdftex,13pt,a4paper]{article}

\usepackage[utf8]{inputenc}
\usepackage[T1]{fontenc}
\usepackage[MeX]{polski}
\usepackage{graphicx}
\newcommand{\HRule}{\rule{\linewidth}{0.5mm}}


\begin{document}

\begin{titlepage}
\begin{center}

\includegraphics[width=0.75\textwidth]{../logo.png}~\\[2cm]

\textsc{\LARGE DOKUMENTACJA UŻYTKOWNIKA}\\[1.5cm]

\textsc{\Large Projekt z Sieci Komputerowych i Bezpieczeństwa}\\[0.5cm]

% Title
\HRule \\[0.4cm]
{ \huge \bfseries Wyszukiwarka połączeń \\[0.4cm] }

\HRule \\[1.5cm]

% Author and supervisor
\begin{minipage}{0.4\textwidth}
\begin{flushleft} \large
\emph{Autor:}\\
Krzysztof \textsc{Barański}
\end{flushleft}
\end{minipage}
\begin{minipage}{0.4\textwidth}
\begin{flushright} \large
\emph{Nadzorujący:} \\
dr~Edward \textsc{Szczypka}
\end{flushright}
\end{minipage}

\vfill

% Bottom of the page
{\large Styczeń 2015}

\end{center}
\end{titlepage}

\tableofcontents

\newpage

\section{Wymagania}

\subsection{Aplikacja kliencka}

\paragraph{System operacyjny}\mbox{}\\
	Linux\\
	Testowany na \texttt{Linux debian 3.2.0-4-amd64}.

\paragraph{Środowisko}\mbox{}\\
	Java SE Development Kit w wersji 1.8\\
	Bash

\paragraph{Sprzęt}\mbox{}\\
	Połączenie z internetem.

\subsection{Aplikacja serwerowa}

\paragraph{System operacyjny}\mbox{}\\
	Linux\\
	Testowany na \texttt{Linux debian 3.2.0-4-amd64}.

\paragraph{Środowisko}\mbox{}\\
	Java SE Development Kit w wersji 1.8\\
	Bash

\paragraph{Sprzęt}\mbox{}\\
	Połączenie z internetem.

\subsection{Aplikacja przewoźnika}

\paragraph{System operacyjny}\mbox{}\\
	Linux\\
	Testowany na \texttt{Linux debian 3.2.0-4-amd64}.

\paragraph{Środowisko}\mbox{}\\
	Java SE Development Kit w wersji 1.8\\
	SQLite3\\
	Bash

\paragraph{Sprzęt}\mbox{}\\
	Połączenie z internetem.
	
\section{Konfiguracja}

\subsection{Aplikacja kliencka}

	Przed uruchomieniem aplikacji należy ją skompilować - służy do tego skrypt \emph{compile} znajdujący się w katalogu \emph{client}. Następnie należy uruchomić skrypt \emph{setup} znajdujący się w tym samym katalogu w celu skonfigurowania aplikacji. Podczas konfiguracji wymagane będzie podanie adresu serwera i portu, na którym serwer nasłuchuje.
  
\subsection{Aplikacja serwerowa}

	Przed uruchomieniem serwera należy go skompilować wywołując skrypt \emph{compile} znajdujący się w katalogu \emph{server}. Następnie należy skonfigurować serwer za pomocą skryptu \emph{setup}. Podczas konfiguracji wymagane będzie podanie portu, na którym będzie nasłuchiwał serwer.
	
	Przewoźników można wyświetlać, dodawać i usuwać za pomocą skryptu \emph{carriers}. Uruchomienie bez żadnych parametrów wyświetla pomoc.
Można to również zrobić ręcznie, edytując plik \emph{server/database/carriers.db}.

\subsection{Aplikacja przewoźnika}

	Przed uruchomieniem aplikacji przewoźnika należy ją skompilować wywołując skrypt \emph{compile} znajdujący się w katalogu \emph{carrier}. Następnie należy skonfigurować aplikację za pomocą skryptu \emph{setup}. Podczas konfiguracji wymagane będzie podanie nazwy przewoźnika ( np. nazwa firmy przewozowej ), numeru portu, na którym będzie nasłuchiwał serwer przewoźnika oraz nazwę bazy danych, która zostanie stworzona w celu przechowywania rozkładu jazdy.
	
	W katalogu \emph{carrier} znajduje się również skrypt \emph{random\_timetable}, który służy do wypełniania bazy danych losowym rozkładem jazdy. Podczas działania, skrypt zapyta o liczbę rekordów do wygenerowania. Skrypt można wywoływać zarówno przed, jak i po skonfigurowaniu aplikacji wielokrotnie - za każdym razem zostanie dodane do bazy tyle rekordów ile zostało podane podczas działania skryptu.

\section{Uruchamianie}

\subsection{Aplikacja kliencka}
	Aby uruchomić skompilowaną i skonfigurowaną aplikację klienta należy w katalogu \emph{client} uruchomić skrypt \emph{run}.

\subsection{Aplikacja serwerowa}
	Aby uruchomić skompilowaną i skonfigurowaną aplikację serwerową należy w katalogu \emph{server} uruchomić skrypt \emph{run}.

\subsection{Aplikacja przewoźnika}
	Aby uruchomić skompilowaną i skonfigurowaną aplikację przewoźnika należy w katalogu \emph{carrier} uruchomić skrypt \emph{run}.

\section{Obsługa}

\subsection{Aplikacja kliencka}
	Po uruchomieniu aplikacji, w oknie konsoli pojawi się logo projektu oraz początek sesji wyszukiwania połączenia.
	
	Po słowie \textbf{FROM:} oraz \textbf{TO:} należy wpisać odpowiednio: miasto początkowe i końcowe podróży, używając wyłącznie liter alfabetu angielskiego, pauzy lub spacji. Lista dostępnych miast znajduje się na końcu dokumentacji.\\
	
	Następnie po słowie \textbf{DATE:} należy wpisać datę początku podróży w formacie \emph{YYYY-MM-DD}, np. \emph{2015-01-26}.\\
	
	Na koniec po słowie \textbf{TIME:} należy wpisać godzinę początku podróży w formacie \emph{HH:MM}, np. \emph{07:56}.\\
	
	Jeżeli istnieje połączenie podane w zapytaniu, aplikacja znajdzie je, a następnie wypisze kolejne odcinki podróży.
	
	Aby wyszukać kolejne połączenie, należy po linii: \textit{"Would you like to find a new connection? If so, type 'yes':"} wpisać słowo \textit{"yes"} bez cudzysłowu. Każde inne słowo, w szczególności pusta linia, kończy działanie aplikacji. 
	
\subsection{Aplikacja serwerowa}
	Obsługa serwera polega jedynie na sprawdzaniu, czy działa. Serwer jest odporny na różnego rodzaju błędy i wyjątki, aczkolwiek może się zdarzyć nieprzewidziany dotąd błąd, po którym aplikacja się zatrzyma. W celu zatrzymania aplikacji, w obecnej wersji należy wysłać jej jakiś mocny sygnał np. SIGINT ( w konsoli zazwyczaj Ctrl+C ).

\subsection{Aplikacja przewoźnika}
	Obsługa aplikacji przewoźnika jest analogiczna do obsługi aplikacji serwera.
	
\section{Odinstalowywanie}

\subsection{Aplikacja kliencka}
W celu odinstalowania (lub też przywrócenia stanu aplikacji do stanu wyjściowego) należy uruchomić skrypt \emph{uninstall} w odpowiednim podkatalogu projektu.

\subsection{Aplikacja serwerowa}
Patrz: Aplikacja kliencka

\subsection{Aplikacja przewoźnika}
Patrz: Aplikacja kliencka

\section{Lista obsługiwanych miast}

Poniżej znajduje się lista miast, które są aktualnie obsługiwane przez wyszukiwarkę:

\begin{itemize}
\item Bialystok
\item Bielsko-Biala
\item Bydgoszcz
\item Bytom
\item Chelm
\item Elblag
\item Gdansk
\item Gliwice
\item Gorzow Wielkopolski
\item Kalisz
\item Katowice
\item Kielce
\item Koszalin
\item Krakow
\item Legnica
\item Lublin
\item Lodz
\item Nowy Sacz
\item Olsztyn
\item Opole
\item Pila
\item Poznan
\item Rzeszow
\item Slupsk
\item Suwalki
\item Szczecin
\item Swinoujscie
\item Tarnow
\item Torun
\item Walbrzych
\item Warszawa
\item Wloclawek
\item Wroclaw
\item Zakopane
\item Zamosc
\item Zielona Gora
\end{itemize}

\end{document}

