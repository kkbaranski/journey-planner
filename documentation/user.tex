\documentclass[pdftex,13pt,a4paper]{article}

\usepackage[utf8]{inputenc}
\usepackage[T1]{fontenc}
\usepackage[MeX]{polski}
\usepackage{graphicx}
\newcommand{\HRule}{\rule{\linewidth}{0.5mm}}


\begin{document}

\begin{titlepage}
\begin{center}

\includegraphics[width=0.75\textwidth]{../logo.png}~\\[2cm]

\textsc{\LARGE DOKUMENTACJA UŻYTKOWNIKA}\\[1.5cm]

\textsc{\Large Projekt z Sieci Komputerowych i Bezpieczeństwa}\\[0.5cm]

% Title
\HRule \\[0.4cm]
{ \huge \bfseries Wyszukiwarka połączeń \\[0.4cm] }

\HRule \\[1.5cm]

% Author and supervisor
\begin{minipage}{0.4\textwidth}
\begin{flushleft} \large
\emph{Autor:}\\
Krzysztof \textsc{Barański}
\end{flushleft}
\end{minipage}
\begin{minipage}{0.4\textwidth}
\begin{flushright} \large
\emph{Nadzorujący:} \\
dr~Edward \textsc{Szczypka}
\end{flushright}
\end{minipage}

\vfill

% Bottom of the page
{\large Styczeń 2015}

\end{center}
\end{titlepage}

\tableofcontents

\newpage

\section{Wymagania}

\subsection{Aplikacja kliencka}

\paragraph{System operacyjny}\mbox{}\\
	Linux\\
	Testowany na \texttt{Linux debian 3.2.0-4-amd64}.

\paragraph{Środowisko}\mbox{}\\
	Java SE Development Kit w wersji 1.8\\
	Bash

\paragraph{Sprzęt}\mbox{}\\
	Połączenie z internetem.

\subsection{Aplikacja serwerowa}

\paragraph{System operacyjny}\mbox{}\\
	Linux\\
	Testowany na \texttt{Linux debian 3.2.0-4-amd64}.

\paragraph{Środowisko}\mbox{}\\
	Java SE Development Kit w wersji 1.8\\
	Bash

\paragraph{Sprzęt}\mbox{}\\
	Połączenie z internetem.

\subsection{Aplikacja przewoźnika}

\paragraph{System operacyjny}\mbox{}\\
	Linux\\
	Testowany na \texttt{Linux debian 3.2.0-4-amd64}.

\paragraph{Środowisko}\mbox{}\\
	Java SE Development Kit w wersji 1.8\\
	SQLite3\\
	Bash

\paragraph{Sprzęt}\mbox{}\\
	Połączenie z internetem.
	
\section{Konfiguracja}

\subsection{Aplikacja kliencka}

	W pliku \emph{client/config.properties} należy ustawić właściwości 	\emph{server\_address} i \emph{server\_port} odpowiednio na adres serwera i port, na którym nasłuchuje serwer. 
  
  Przykład: 
  
  \emph{server\_address=localhost}  
  
  \emph{server\_port=9911}
  
\subsection{Aplikacja serwerowa}

	W pliku \emph{server/config.properties} należy ustawić właściwość \emph{server\_port} na port, na którym nasłuchuje serwer. 
  
  Przykład: 
  
  \emph{server\_port=9911}

\subsection{Aplikacja przewoźnika}

	W pliku \emph{carrier/config.properties} należy ustawić właściwość \emph{server\_port} na port, na którym nasłuchuje serwer. 
  
  Przykład: 
  
  \emph{server\_port=9911}

\section{Uruchamianie}

\subsection{Aplikacja kliencka}
todo

\subsection{Aplikacja serwerowa}
todo

\subsection{Aplikacja przewoźnika}
todo

\end{document}

